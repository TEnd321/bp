\chapter{Dokumentácia protokolu}

Peer-to-peer sieť na zdieľanie novinových článkov implementuje peer-to-peer protokol na zdieľanie novinových článkov, skrátene \npps{}, z anglického ,,Newspaper P2P Sharing Protocol``.

\section{Prehľad fungovania protokolu}

Sekcia opisuje neformálny popis sieťového protokolu \npps{} a jeho fungovania. Sieť používajúca protokol je vytváraná pre jedny konkrétne noviny a neslúži na komunikáciu medzi dvoma rôznymi novinami, aj keď taká komunikácia je síce možná, znamenalo by to výmenu informácií medzi dvomi oddelenými a okrem protokolu ničím nesúvisiacimi sieťami.

Sieťový protokol \npps{} je decentralizovaný protokol postavený na princípe peer-to-peer. Základnou stavebnou jednotkou celej siete je takzvaný peer. Hoci sa jedná o peer-to-peer protokol, nie je úplná decentralizácia možná. Súvisí to s tým, že inštitúcia novín samotná má nejakú hierarchiu a teda nie je pravdou, že by dvaja peeri boli nutne na rovnakej úrovni. Tak tomu je z uhla pohľadu používateľa. Čo sa týka protokolu samotného, má každý peer rovnaké možnosti.

Peer môže byť ako čitateľom, tak aj novinárom či šéfredaktorom novín. Tieto noviny si peer vytvára sám, pričom všetky články, ktoré do novín on sám uverejní, budú k dispozícii verejne pre každého peera, ktorý si o ne požiada. Vyžadovaný je formát markdown, pričom je možné podporovať aj ďalšie spôsoby formátovania textu, ako napríklad HTML, TeX, či roff.

Na identifikáciu článkov sa používa hash článku, pričom ten by mal byť spočítaný z celého článku, z jeho ako textovej, tak aj multimediálnej časti. Podpora obrázkov tvorí minimálnu požiadavku pre podporu multimédií. Možné, a aj odporúčané rozšírenie je o podporu videa, či audia.

Peer posiela správy po sieti buď to priamo druhému peerovi, ktorý je spolu s ním na lokálnej sieti alebo má verejnú IPv4 adresu. Ak sa však jeden z peerov nachádza za smerovačom, ktorý má zapnutý NAT, je potrebné tento NAT obísť, pričom sa na tento účel používa dvojica už existujúcich protokolov STUN a TURN. Peer sťa by STUN klient naviaže spojenie prostredníctvom STUN servera aby sa pripojil k ďalšiemu STUN klientovi, teda k ďalšiemu peerovi. Tento postup však v prípade niektorých implementácií NAT nie je možný (napríklad v prípade symetrického NAT-u) a teda je potrebné, aby všetká komunikácia medzi dvoma peermi tiekla skrz jednu konkrétnu sieťovú entitu. K tomu slúži rozšírenie STUN protokolu pod názvom TURN. STUN server implementujúci TURN sa stane TURN serverom a bude sprostredkúvať komunikáciu medzi TURN klientom a TURN peerom, čo je \npps{} peer, ku ktorému sa chceme pripojiť. 

V \npps{} sieti je TURN serverom pre dané noviny minimálne ich šéfredaktor. Bolo by možné nasadiť ďalších \npps{} peerov, ktorí by plnili funkciu TURN serveru, ak by to pre plynulejší chod siete bolo potrebné. 

Správa je do siete serializovaná pomocou Google Protocol buffers, ktoré majú k dispozícii API pre viaceré programovacie jazyky a prostredia. Takto serializovaná správa je následne zašifrovaná symetrickým kľúčom a odoslaná po sieti jej príjemcovi. Ten si ju následne rozšifruje a spracuje. Potom čo je správa zašifrovaná ale predtým, čo je odoslaná je potrebné pridať na začiatok správy takzvaný identifikačný bajt, ktorý identifikuje typ správy, teda či sa jedná o štandardnú správu alebo o správu obsahujúcu symetrický kľúč počas jeho výmeny pri (väčšinou) prvej komunikácii jedného peera s tým druhým. Pridaním tohto bajtu vznikne takzvaná metaspráva. Ďalšie typy metaspráv je možné kedykoľvek pridať. 

Správy sú šifrované symetricky, pričom každý peer má uložený spoločný symetrický kľúč s ďalším peerom. Ak jeden z peerov tento kľúč nejakým spôsobom stratí, je potrebné, aby sa opäť vygeneroval, podobným spôsobom, ako pri prvom ,,stretnutí sa`` dvoch peerov. Počas tejto výmeny symetrických kľúčov sa použije špeciálny typ metasprávy a, keďže symetrický kľúč ešte nemajú obe strany, je potrebné, aby bola táto správa šifrovaná asymetricky, pomocou verejných kľúčov oboch strán. Najprv však dôjde k podpisu správy a až tak k jej asymetrickému zašifrovaniu. Používajú sa metódy autentifikovaného šifrovania (tzv. AE, Authenticated Encryption) a teda ak by mali byť dáta nejakým spôsobom pozmenené počas prenosu, bude táto zmena zistená a ohlásená užívateľovi. Multimediálne prvky môžu byť buď to zašifrované tiež, alebo ak nie, tak sú aspoň rovnako autentifikované ako textové dáta. Na tento účel slúžia rôzne schémy typu AEAD, čo je skratka z anglického Authenticated Encryption with Additional Data, umožňujúce sledovať, či neboli nešifrované dáta počas prenosu nijak pozmenené. 

